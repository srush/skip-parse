\documentclass{article}
% \usepackage[pdftex,active,tightpage]{preview}
%\setlength\PreviewBorder{2mm} % use to add a border around the image
\usepackage{tikz}
\usepackage{proof}
\usetikzlibrary{arrows}
\usetikzlibrary{decorations.markings}
\tikzset{
  >=latex,text height=1.5ex,text depth=0.25ex
}

\begin{document}
% \begin{preview}

This note describes a new algorithm for sentence compression using dependency parsing.

\newcommand{\abox}{\scalebox{0.2}{\tikz{
    \coordinate (A) at (0,0);
    \coordinate (B) at (0,1.5cm);
    \coordinate (C) at (-2cm, 1.5cm);
    \coordinate (D) at (-2cm, 0cm);
    \draw[line width = 0.05cm] (A)--(B)--(C)--(D)--cycle;
    }}}

\newcommand{\rtrap}{\scalebox{0.2}{\tikz{
    \coordinate (A) at (0,0);
    \coordinate (B) at (90:1.5cm);
    \coordinate (C) at (2.5,0.9cm);
    \coordinate (D) at (0:2.5cm);
    \draw[line width = 0.05cm] (A)--(B)--(C)--(D)--cycle;
    }}}


\newcommand{\ltrap}{\scalebox{0.2}{\tikz{
    \coordinate (A) at (0,0);
    \coordinate (B) at (90:1.5cm);
    \coordinate (C) at (-2.5,0.9cm);
    \coordinate (D) at (180:2.5cm);
    \draw[line width = 0.05cm] (A)--(B)--(C)--(D)--cycle;
    }}}

% \newcommand{\rtrapskip}{\scalebox{0.2}{\tikz{
%     \coordinate (A) at (0,0);
%     \coordinate (B) at (90:1.5cm);
%     \coordinate (C) at (2.5,0.9cm);
%     \coordinate (D) at (0:2.5cm);
%     \draw[line width = 0.05cm] (A)--(B)--(C)--(D)--cycle;
%     }}}

\newcommand{\rtrapskip}{\scalebox{0.2}{\tikz{
      \begin{scope}[decoration={
          markings,
          mark=at position 0.3 with {\arrow[scale=1.75]{<}},
          mark=at position 0.55 with {\arrow[scale=1.75]{<}},
          mark=at position 0.8 with {\arrow[scale=1.75]{<}}
        }]
        \coordinate (A) at (0,0);
        \coordinate (B) at (90:1.5cm);
        \coordinate (C) at (2.5,0.9cm);
        \coordinate (D) at (0:2.5cm);
        \draw[postaction={decorate}, line width = 0.05cm] (B) -- (C);
        \draw[line width = 0.05cm] (A)--(B)--(C)--(D)--cycle;
      \end{scope}
    }}}

\newcommand{\rtri}{\scalebox{0.2}{\tikz{
    \coordinate (A) at (0,0);
    \coordinate (B) at (90:1.5cm);
    \coordinate (C) at (180:-1.7cm);
    \draw[line width = 0.05cm] (A)--(B)--(C)--cycle;
    }}}

\newcommand{\ltri}{\scalebox{0.2}{\tikz{
    \coordinate (A) at (0,0);
    \coordinate (B) at (90:1.5cm);
    \coordinate (C) at (180:1.7cm);
    \draw[line width = 0.05cm] (A)--(B)--(C)--cycle;
    }}}


\section{Standard First-Order Dependency Parsing }

The standard algorithm for first order dependency parsing consists of the following rules.

\textbf{Premise:}
\[(\rtri, i,i), (\ltri, i,i)\ \ \  \forall i \in \{0 \ldots n\}\]

\textbf{Rules:}


\begin{eqnarray*}
\infer[(i \rightarrow j)]{(\rtrap, i,j)}{(\rtri, i,k)  &  (\ltri, k+1, j) } &  \forall i\leq k < j \\\\
\infer[(j \rightarrow i)]{(\ltrap, i,j)}{(\rtri, i,k)  &  (\ltri, k+1, j) } & \forall i\leq k < j \\\\
\infer{(\rtri, i,j)}{(\rtrap, i,k)  &  (\rtri, k, j) }    &  \forall i<  k \leq j \\\\
\infer{(\ltri, i,j)}{(\ltri, i,k)  &  (\ltrap, k, j) }  & \forall i\leq k < j \\\\
\end{eqnarray*}

\textbf{Goal:}\[ (\rtri, 0,n)\]



\section{Skip Parsing}


\textbf{Premise:}
\[(\rtri, i,i), (\ltri, i,i)\ \ \  \forall i \in \{0 \ldots n\}\]

\textbf{Rules:}


\begin{eqnarray*}
\infer[(i \rightarrow j)]{(\rtrap, i,j)}{(\rtri, i,k)  &  (\ltri, k+1, j) } &  \forall i\leq k < j \\\\
\infer[(j \rightarrow i)]{(\ltrap, i,j)}{(\rtri, i,k)  &  (\ltri, k+1, j) } & \forall i\leq k < j \\\\
\infer{(\rtri, i,j)}{(\rtrap, i,k)  &  (\rtri, k, j) }    &  \forall i<  k \leq j \\\\
\infer{(\ltri, i,j)}{(\ltri, i,k)  &  (\ltrap, k, j) }  & \forall i\leq k < j \\\\
\infer[\mathrm{skip}(k)]{(\rtrapskip, i,j)}{(\rtri, i,k-1)  &  (\ltri, k, k) } &  \forall i < k \\\\
\infer{(\rtri, i,j)}{(\rtrapskip, i,k)  &  (\rtri, k, k) } &  \forall i < k \\\\
\end{eqnarray*}

\textbf{Goal:}\[ (\rtri, 0,n)\]

\section{Skip Bigram Parsing}

In this styles of parsing


\textbf{Premise:}
\[(\rtri, i,i, i), (\ltri, i,i, i)\ \ \  \forall i \in \{0 \ldots n\}\]

\textbf{Rules:}


\begin{eqnarray*}
\infer{(\rtri, i,i, p)}{(\rtri, i, i,i)} &  \forall i,k,  0 \leq i < p \leq n + 1\\\\
\infer{(\rtrap, i,j, p)}{(\rtri, i,k, k+1)  &  (\ltri, k+1, j, p) } &  \forall 0 \leq i\leq k < j < p \\\\
\infer{(\ltrap, i,j, p)}{(\rtri, i,k, k +1)  &  (\ltri, k+1, j, p) } & \forall i\leq k < j <p\\\\
\infer{(\rtri, i,j, p)}{(\rtrap, i,k, k)  &  (\rtri, k, j, p) }    &  \forall i<  k \leq j < p \\\\
\infer{(\ltri, i,j, p)}{(\ltri, i,k, k)  &  (\ltrap, k, j, p) }  & \forall i\leq k < j \\\\
\infer{(\rtrapskip, i,j, p)}{(\rtri, i,k-1, p)  &  (\ltri, k, k) } &  \forall i < k < p\\\\
\infer{(\rtri, i,j, p)}{(\rtrapskip, i,k, p)  &  (\rtri, k, k, k) } &  \forall i < k < p\\\\
\end{eqnarray*}

\textbf{Goal:}\[ (\rtri, 0,n, n + 1)\]


\section{Second-Order Dependency Parsing}

\textbf{Premise:}
\[(\rtri, i,i), (\ltri, i,i)\ \ \  \forall i \in \{0 \ldots n\}\]

\textbf{Rules:}

\begin{eqnarray*}
\infer{(\abox, i,j)}{(\rtri, i,k)  &  (\ltri, k+1, j) } &  \forall i\leq k < j \\\\
\infer[(i \rightarrow j)]{(\rtrap, i,j)}{(\rtri, i,i)  &  (\ltri, i+1, j) } &  \forall i < j \\\\
\infer[(j \rightarrow i)]{(\ltrap, i,j)}{(\rtri, i,j-1)  &  (\ltri, j, j) } & \forall i < j \\\\
\infer[(i \rightarrow k\ j)]{(\rtrap, i,j)}{(\rtrap, i,k)  &  (\abox, k, j) } &  \forall i\leq k < j \\\\
\infer[(j \rightarrow k\ i)]{(\ltrap, i,j)}{(\abox, i,k)  &  (\ltrap, k, j) } &  \forall i\leq k < j \\\\
\infer{(\rtri, i,j)}{(\rtrap, i,k)  &  (\rtri, k, j) }    &  \forall i<  k \leq j \\\\
\infer{(\ltri, i,j)}{(\ltri, i,k)  &  (\ltrap, k, j) }  & \forall i\leq k < j \\\\
\end{eqnarray*}

\textbf{Goal:}\[ (\rtri, 0,n)\]




% \begin{figure}
% \begin{tikzpicture}[node distance=0.5cm]
% \tikzstyle{every node}=[font=\large]
% \tikzstyle{plus}=[xshift=0.2cm, font=\large]
% \tikzstyle{equal}=[xshift=0.2cm,  font=\large]


%   \begin{scope}
%     \node [xshift=-1cm, yshift=0.7cm]{(a)};
%   \begin{scope}[]
%     \coordinate (A) at (0,0);
%     \coordinate (B) at (90:1.5cm);
%     \coordinate (C) at (2.5,0.9cm);
%     \coordinate (D) at (0:2.5cm);
%     \draw[line width = 0.05cm] (A)--(B)--(C)--(D)--cycle;
%     \node (h) [below of = A] {$h$};
%     \node (m) [below of = D] {$m$};
%     \path [draw, bend angle = 30, shorten <= 0.1cm, shorten >= 0.1cm, line width = 0.05cm] (A) edge [bend left, ->] node [above] {} (D);
%     \node [right of = D, yshift=0.7cm, equal]{$\leftarrow$};
%     \node [yshift=0.6cm, xshift=-0.3cm]{\textbf{I}};
%     \node{$\forall h, m, r$ };
%   \end{scope}



%   \begin{scope}[xshift = 4cm]
%     \coordinate (A) at (0,0);
%     \coordinate (B) at (90:1.5cm);
%     \coordinate (C) at (180:-1.7cm);
%     \draw[line width = 0.05cm] (A)--(B)--(C)--cycle;
%     \node [below of = A] {$h$};
%     \node [below of = C] {$r$};
%     \node [right of = C, yshift=0.7cm, plus]{$+$};
%     \node [yshift=0.6cm, xshift=-0.3cm]{\textbf{C}};
%   \end{scope}




%   \begin{scope}[xshift = 8.8cm]
%     \coordinate (A) at (0,0);
%     \coordinate (B) at (90:0.9cm);
%     \coordinate (C) at (180:1.75cm);
%     \draw[line width = 0.05cm] (A)--(B)--(C)--cycle;
%     \node [below of = A] {$m$};
%     \node [below of = C] {$r+1$};
%     \node [yshift=0.6cm, xshift=-1.8cm]{\textbf{C}};
%   \end{scope}
%   \end{scope}


%   \begin{scope}[yshift=-3cm]
%     \node [xshift=-1cm, yshift=0.7cm]{(b)};
%   \begin{scope}
%     \coordinate (A) at (0,0);
%     \coordinate (B) at (90:1.5cm);
%     \coordinate (C) at (180:-2.5cm);
%     \draw[line width = 0.05cm] (A)--(B)--(C)--cycle;
%     \node [below of = A] {$h$};
%     \node [below of = C] {$e$};
%     \node [right of = C, yshift=0.7cm, equal]{$\leftarrow$};
%     \node [yshift=0.6cm, xshift=-0.3cm]{\textbf{C}};
%   \end{scope}

%   \begin{scope}[xshift = 4cm]
%     \coordinate (A) at (0,0);
%     \coordinate (B) at (90:1.5cm);
%     \coordinate (C) at (1.7,0.9cm);
%     \coordinate (D) at (0:1.7cm);
%     \draw[line width = 0.05cm] (A)--(B)--(C)--(D)--cycle;
%     \node [below of = A] {$h$};
%     \node [below of = D] {$m$};
%     \path [draw, bend angle = 50, shorten <= 0.1cm, shorten >= 0.1cm,
%     line width = 0.05cm] (A) edge [bend left, ->] node [above] {} (D);
%     \node [right of = D, yshift=0.7cm, plus]{$+$};
%     \node [yshift=0.6cm, xshift=-0.3cm]{\textbf{I}};
%   \end{scope}


%   \begin{scope}[xshift = 7.3cm]
%     \coordinate (A) at (0,0);
%     \coordinate (B) at (90:0.9cm);
%     \coordinate (C) at (180:-1.75cm);
%     \draw[line width = 0.05cm] (A)--(B)--(C)--cycle;
%     \node [below of = A] {$m$};
%     \node [below of = C] {$e$};
%     \node [yshift=0.6cm, xshift=-0.3cm]{\textbf{C}};
%   \end{scope}
%   \end{scope}

\section{Computing with Fixed m}





% \end{tikzpicture}
% \end{figure}
% \end{preview}
\end{document}

%%% Local Variables:
%%% mode: latex
%%% TeX-master: t
%%% End:
