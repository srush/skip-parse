\documentclass[11pt,a4paper]{article}
\usepackage{acl2014}
\usepackage{times}
\usepackage{url}
\usepackage{latexsym}

\usepackage{amsmath,amsfonts,amssymb}
\usepackage[mathscr]{euscript}
\usepackage{algorithm}
\usepackage[noend]{algpseudocode}
\usepackage{multirow}
\usepackage{tikz}

% Actually apply letter size for PDFs; letterpaper just scales text -K
\setlength{\pdfpagewidth}{8.268in}
\setlength{\pdfpageheight}{11.693in}

% argmax and argmin - K
\DeclareMathOperator*{\argmax}{arg\,max}
\DeclareMathOperator*{\argmin}{arg\,min}

\newenvironment{packed_enumerate}{
\begin{enumerate}
  \setlength{\itemsep}{1pt}
  \setlength{\parskip}{0pt}
  \setlength{\parsep}{0pt}
}{\end{enumerate}}
\newenvironment{packed_itemize}{
\begin{itemize}
  \setlength{\itemsep}{1pt}
  \setlength{\parskip}{0pt}
  \setlength{\parsep}{0pt}
}{\end{itemize}}

\setlength\titlebox{6.5cm}    % Expanding the titlebox

\title{Efficient Multi-Structure Sentence Compression}
\title{Parsing with Skips for Sentence Compression}
\title{Optimal and Efficient Multi-Structure Compression}
\title{Compressive Parsing}
\title{First and Second-Order Compressive Parsing}
\title{Efficient Compressive Parsing for Sentence Compression}

%\author{Kapil Thadani \\
%        Department of Computer Science\\
%	     Columbia University\\
%	     New York, NY 10025, USA\\
%        {\tt kapil@cs.columbia.edu} \And
%        Alexander M.~Rush \\
%        MIT CSAIL \\
%        Cambridge, MA 02139, USA \\
%        {\tt srush@csail.mit.edu } \\}
\author{}

\date{}

\begin{document}
\maketitle
\begin{abstract}
    Supervised approaches to sentence compression have previously been shown
    to benefit when the dependency structure of the output sentence is
    included in the inference objective, but prior formulations of this
    inference task do not admit efficient algorithms.
%    and require integer programming or approximate solutions.
    We present a polynomial-time approach to recover dependency trees
    for compressed sentences in which dynamic programming is used to
    produce a valid parse tree over a subset of the tokens in the input
    in order to satisfy a compression rate. In addition to first-order
    dependency structure, this approach also recovers
    second-order dependencies as well as a bigram factorization of the
    output sentence with no asymptotic overhead.
    Experiments on two standard corpora for sentence compression show
    improved performance against previous work

%    Sentence compression has been shown to benefit from
%    joint inference involving both n-gram and dependency-factored objectives
%    but this typically requires expensive integer programming.
%    We explore instead the use of Lagrangian relaxation
%    to decouple the two subproblems and solve them separately.
%    While dynamic programming is viable for bigram-based sentence compression,
%    finding optimal compressed trees within graphs is NP-hard.
%    We recover approximate solutions to this problem using
%    LP relaxation and maximum spanning tree algorithms, yielding
%    techniques that can be combined with the efficient
%    bigram-based inference approach using Lagrange multipliers.
%    Experiments show that these approximation strategies produce results
%    comparable to a state-of-the-art integer linear programming formulation
%    for the same joint inference task along with a significant improvement
%    in runtime.
\end{abstract}

\section{Introduction}
\label{intro}

Text-to-text generation problems such as paraphrase generation,
sentence fusion and text simplification involve
monolingual transformations of text for specific goals
such as reducing verbosity or assisting comprehension.
The most popular of these problems is the task of
sentence \emph{compression} which involves
the production of well-formed output sentences that are shorter than
their corresponding input sentences while nevertheless preserving their
meaning.
Compression problems have received significant attention in
recent years due to their usefulness in document summarization~\cite{zajic07}
as well as the increasing number of sources of compression
data~\cite{knight00,clarke06a,galanis11,filippova13} for training
and evaluation.

%While sentence compression is often
%incorporated into systems for document
%summarization~\cite{daume02,zajic07,clarke07,martins09a,bergkirkpatrick11,woodsend12,almeida13,molina13,li13,qian13}, the standalone compression task
%has also received significant attention,
%in part due to the availability of parallel sentence corpora such as
%the Ziff-Davis corpus~\cite{knight00} and the Edinburgh compression
%corpora~\cite{clarke06a}.

Sentence compression is usually formulated as a \textit{token deletion}
problem\footnote{Following the terminology used for text
    summarization, this setting is also referred to as \emph{extractive}
    compression by \newcite{cohn08a} and \newcite{galanis10}.}
in which the compressed output is synthesized using only the
tokens in the input sentence without any reordering or paraphrasing,
as seen in the following example from the corpus of
\newcite{clarke06a}.

\begin{tabular}{p{190pt}}
    \small\vspace{0pt}
\textbf{Original:} In 1967 Chapman, who had cultivated a
    conventional image with his ubiquitous tweed jacket and pipe, by his own
    later admission stunned a party attended by his friends and future Python
    colleagues by coming out as a homosexual.\\[4pt]
        \small
\textbf{Compressed:} In 1967 Chapman, who had cultivated a
conventional image, stunned a party by coming out as a homosexual.\\[4pt]
\end{tabular}
%Compressed sentences are typically generated by inference algorithms
%that target some structured representation of output sentences.
While a wide variety of structured inference techniques have been
proposed to
assemble compressed sentences in this setting, the production of
compressed dependency trees is prevalent in most recent work for
standalone sentence
compression~\cite{filippova08a,nomoto09,galanis10,thadani13a,filippova13,thadani14}
as well as joint compression and
summarization~\cite{martins09a,almeida13,qian13}. However, each of these
approaches is limited by one or more of the following: (i) a restriction
that the output tree can only use edges from the input parse,
resulting in unreachable gold compressions and increased sensitivity to
parser errors,
(ii) an intractable formulation which necessitates the use of expensive
integer linear programs (ILPs) or algorithms for approximate solutions,
and (iii) a reliance on hand-crafted rules or constraints which can be
brittle and domain-dependent.

In this work, we present polynomial-time algorithms for
\emph{compressive parsing} which generate optimal dependency trees
from all possible parse trees over output compressions without any need for
heuristics or hand-crafted rules.
These dynamic programming strategies extend the well-known Eisner
algorithm for projective parsing~\cite{eisner96}
in order to drop a fixed or variable number of tokens
in the output parse tree.
Our formulation also scores a bigram
factorization of the compressed sentence with no asymptotic overhead,
thereby offering an efficient alternative to the ILP for joint n-gram and
dependency structured inference proposed by \newcite{thadani13a}.
%\footnote{
%    \newcite{thadani13a} produce \emph{non-projective} trees which cannot
%    be recovered efficiently~\cite{lau06} and must
%    consequently rely on ILPs or LP relaxations~\cite{thadani14}.}
Finally, this dynamic program can easily be extended to richer
second-order compressive parsing \cite{eisner96,mcdonald06b} in
which scores can be defined over consecutive parsing decisions without
any increase in runtime.

The contributions of this paper include:
\begin{packed_itemize}
\item An $O(n^3)$ time dynamic programming algorithm to jointly recover the
    optimal compressed dependency tree and bigram factorization over an
    input sentence of length $n$ when no compression rate is specified.
\item An $O(n^3m^2)$ time algorithm to recover the optimal compressed
    tree and bigram sequence covering exactly $m < n$ tokens.
\item A bisection-based first-pass strategy to decrease runtime
    by recovering a fraction of compressions with $m$ tokens in
    $O(n^3k)$ where $k << m^2$.
\end{packed_itemize}

\section{Parsing for Compression Inference}
\label{parsing}

A diverse array of inference strategies have been employed in recent
years towards standalone sentence compression and compression within
extractive summarization. In particular, the production of
arc-factored dependency trees to represent compressed sentences
has gained appeal by balancing the straightforwardness
of inference approaches for n-gram subsequences~\cite{mcdonald06a,clarke08}
and the syntactic perspective of tree substitution
grammars~\cite{cohn09,woodsend12} while performing competitively in
experimental evaluations~\cite{galanis10,qian13,thadani14}.

Many recent techniques produce compressed dependency trees by
pruning edges from a dependency parse of the input
sentence or transformations thereof. However, this problem is NP-hard
in the general
case\footnote{A polynomial-time algorithm is viable when the edge weights
    are integers~\cite{lau06}.}
and has consequently been addressed with expensive integer linear
programs~\cite{filippova08a,filippova13} or approximations based on
reranking~\cite{nomoto09,galanis10}. Furthermore, a strong reliance
on the input parse makes these techniques sensitive to parse errors
as well as prone to unreachable reference compressions.

Further constraining the head of the output tree to match that of the
input permits only subtrees to be dropped from the input parse; in this case,
the best compressed tree can be found in linear time using the Viterbi
algorithm for trees. This approach is favored by joint compression
and summarization
systems~\cite{martins09a,gillick09,bergkirkpatrick11,almeida13,qian13}
but the head restriction further exacerbates the fraction of unreachable
reference sentences in a standalone sentence compression setting.
Many of these approaches also define hand-crafted rules
to ensure that important subtrees like the subject of the head aren't
removed.

The joint n-gram and dependency compression approach of
\newcite{thadani13a} does not restrict output parse trees to lie
within input parses but inference remains NP-hard~\cite{thadani14}
because the objective is defined over non-projective trees.
However, since tokens cannot be reordered in extractive sentence
compression, the recovery of linguistically-motivated\footnote{English
    is mostly projective and even
    canonical non-projective languages such
    as Czech, Danish and Turkish have a low rate (1-2\%) of
    non-projective arcs in their respective treebanks~\cite{nivre05}.}
\emph{projective} trees becomes tractable. With the exception of
non-projectivity, the dynamic programming approach described in
the following section fully generalizes over these joint models.

The rest of this section is organized as follows:


\subsection{Features and Learning}
\label{features}

We based the features for our discriminative model on the features
described by \newcite{mcdonald06a} and \newcite{thadani13a}.

Dependency features $\Omega$ included the probability of an arc under
a smoothed dependency grammar derived from the Penn Treebank, the
\emph{fidelity} of the output arc---an indicator of whether it is present in
the input parse---conjoined with its dependency labels, and conjunctions
of the following features: arc fidelity, fidelity of ancestral relations
binned by the distance of the ancestor, direction of attachment,
part-of-speech (POS) tags of the tokens incident to the edge, labels of
constituent chunks containing these tokens and an indicator of whether
they lie in the same chunk.

We also added token-specific features over the dependents of each
arc. These included POS tag sequences of length at most 3 around the token,
the label of the dependency arc to the token in the
input parse conjoined with its POS tag, indicators for
capitalized words and words in parentheses as well as lexical features
covering verb stems, symbols and negations.

Second-order features were comprised of the fidelity of the
second-order dependency conjoined with its label in the input parse.
Although we experimented with a wide range of second-order features on
a development dataset, we found that they were prone to overfitting,
likely due to the small size of the \newcite{clarke08}
datasets used in our evaluation.\footnote{We
    expect these features to be valuable when training on larger
    compression datasets such as the one proposed by
    \newcite{filippova13}.}

Finally, bigram features $\theta$ include the likelihood under a
language model (LM) contructed from the Gigaword corpus, the coarse and
fine POS tags of its tokens and the labels of dependency arcs incident to
these words in the input sentence.

For the experiments in the following section, we trained models using
a variant of the structured perceptron \cite{collins02}
which incorporates minibatches~\cite{zhao13} for easy parallelization and
faster convergence.\footnote{
    We used a minibatch size of 4 in all experiments.}
Overfitting was avoided by averaging parameters and monitoring performance
against a held-out development set during training.
%We followed \newcite{martins09c} in using LP-relaxations
%when training models with ILP inference,
%assuming algorithmic separability~\cite{kulesza07}
%for these compression problems.

\section{Experiments}
\label{experiments}

This section describes the experimental setup used to evaluate the
utility and performance of our proposed inference approach.

\subsection{Datasets}
\label{data}
Compression experiments were conducted over the standard newswire (NW) and
broadcast news transcription (BN) corpora compiled by \newcite{clarke08}.
These datasets include gold compressions which were produced by human
annotators
%---3 in the case of the BN corpus---
who were restricted to only
delete tokens.
We filtered the instances as per \newcite{thadani13a} and used the
same training/dev/test splits as \newcite{clarke08},
resulting in 953/63/603 instances
for the NW corpus and 880/78/404 for the BN corpus.
Reference dependency parses were approximated by parsing reference
compressions with the Stanford dependency
parser.\footnote{\texttt{http://nlp.stanford.edu/software/}}
Following \newcite{napoles11a}, which revealed a strong correlation between
system compression rate and human judgments of compression quality,
all systems were constrained to produce compressed output at a fixed
compression rate determined by the the median reference compressions
for each instance.

\subsection{Evaluation measures}
\label{measures}
We evaluated system performance using F$_1$ metrics over n-grams and
dependencies---produced by parsing
system output with
RASP~\cite{briscoe06} and the Stanford parser---following previous
work~\cite{unno06,clarke08,martins09a,napoles11a,thadani13a,thadani14}.
In the case of the BN corpus which has 3 reference compressions per instance,
F$_1$ scores were averaged over all references.
%, which is frequently used in compression evaluations.
%All ILPs were solved using
%Gurobi,\footnote{\texttt{http://www.gurobi.com}} a high-performance
%commercial-grade solver.
%---to ensure that the
%reported differences between the systems under study are meaningful.

\begin{table*}[t]
    \setlength{\tabcolsep}{4pt}
    \setlength{\belowcaptionskip}{-3pt}
\centering
\begin{tabular}{|p{40pt}l|cccc|ccc|c|}
        \hline
        \multicolumn{2}{|c|}{Inference}  & \multicolumn{4}{c|}{n-grams\ \ F$_1\%$} & \multicolumn{3}{c|}{Syntactic relations F$_1$\%} & Inference \\
        objective & \quad technique & $n=1$ & $2$ & $3$ & $4$ & $\mathbf{z}$ & Stanford & RASP & time (s) \\
        \hline
        \hline
        \multirow{2}{*}{n-grams} & 3-LM (CL08) & 74.96 & 60.60 & 46.83 & 38.71 & - & 60.52 & 57.49 & 0.72 \\
        & DP (McD06) & 78.80 & 66.04 & 52.67 & 42.39 & - & 63.28 & 57.89 & 0.01  \\
        \hline
        \multirow{2}{*}{deps} & LP$\rightarrow$MST   & {\bf 79.61} & 64.32 & 50.36 & 40.97 & 66.57 & 66.82 & 59.70 & 0.07 \\
        & ILP-Dep   & {\bf 80.02} & 65.99 & 52.42 & 43.07 & {\bf 72.43} & 67.63 & 60.78 & 0.16  \\
        \hline
        & DP + LP$\rightarrow$MST   & 79.50 & 66.75 & 53.48 & 44.33 & 64.63 & 67.69 & 60.94 & 0.24 \\
        joint & DP + LP & 79.10 & {\bf 68.22} & {\bf 55.05} & {\bf 45.81} & 65.74 & {\bf 68.24} & {\bf 62.04} & 0.12 \\
        & ILP-Joint (TM13)  & {\bf 80.13} & {\bf 68.34} & {\bf 55.56} & {\bf 46.60} & {\bf 72.57} & {\bf 68.89} & {\bf 62.61} & 0.31  \\
        \hline
\end{tabular}
\caption{Experimental results for the BN corpus, averaged over 3
    gold compressions per instance.
    All systems were restricted to compress to the size of the
    median gold compression yielding an average compression rate of $77.26\%$.}
%    Boldfaced entries indicate significant differences
%    ($p < 0.0005$) under the paired t-test and Wilcoxon's signed rank test.}
\label{tab:bn}
\end{table*}


\begin{table*}[t]
    \setlength{\tabcolsep}{4pt}
    \setlength{\belowcaptionskip}{-5pt}
\centering
\begin{tabular}{|p{40pt}l|cccc|ccc|c|}
        \hline
        \multicolumn{2}{|c|}{Inference}  & \multicolumn{4}{c|}{n-grams\ \ F$_1\%$} & \multicolumn{3}{c|}{Syntactic relations F$_1$\%} & Inference \\
        objective & \quad technique & $n=1$ & $2$ & $3$ & $4$ & $\mathbf{z}$ & Stanford & RASP & time (s) \\
        \hline
        \hline
        \multirow{2}{*}{n-grams} & 3-LM (CL08) & 66.66 & 51.59 & 39.33 & 30.55 & - & 50.76 & 49.57 & 1.22 \\
        & DP (McD06) & 73.18 & 58.31 & 45.07 & 34.77 & - & 56.23 & 51.14 & 0.01  \\
        \hline
        \multirow{2}{*}{deps} & LP$\rightarrow$MST   & 73.32 & 55.12 & 41.18 & 31.44 & 61.01 & 58.37 & 52.57 & 0.12 \\
        & ILP-Dep   & {\bf 73.76} & 57.09 & 43.47 & 33.44 & {\bf 65.45} & 60.06 & 54.31 & 0.28  \\
        \hline
        & DP + LP$\rightarrow$MST   & 73.13 & 57.03 & 43.79 & 34.01 & 57.91 & 58.46 & 53.20 & 0.33 \\
        joint & DP + LP & 72.06 & {\bf 59.83} & {\bf 47.39} & {\bf 37.72} & 58.13 & 58.97 & 53.78 & 0.21 \\
        & ILP-Joint (TM13)  & {\bf 74.00} & {\bf 59.90} & {\bf 47.22} & {\bf 37.01} & {\bf 65.65} & {\bf 61.29} & {\bf 56.24} & 0.60  \\
        \hline
\end{tabular}
\caption{Experimental results for the NW corpus with
    all systems compressing to the size of the
    gold compression, yielding an average compression rate of $70.24\%$.
    In both tables, bold entries show significant gains within a column
    under the paired t-test ($p < 0.05$) and Wilcoxon's signed rank test ($p < 0.01$).}
\label{tab:nw}
\end{table*}

\subsection{Systems}
\label{systems}
We report results over the following systems grouped into three
categories of models:
tokens + n-grams, tokens + dependencies, and joint models.
\begin{packed_itemize}
\item \textbf{3-LM:} A reimplementation of the unsupervised ILP
    of \newcite{clarke08} which infers order-preserving trigram variables
    parameterized with log-likelihood under an LM and
    a significance score for token variables inspired by
    \newcite{hori04}, as well as various linguistically-motivated
    constraints to encourage fluency in output compressions.
\item \textbf{DP:} The bigram-based dynamic program of
    \newcite{mcdonald06a} described in \S\ref{bigram}.\footnote{For
        consistent comparisons with the
        other systems, our reimplementation does not include the
        $k$-best inference strategy presented in \newcite{mcdonald06a}
        %for large margin learning via MIRA.}
        for learning with MIRA.}
\item \textbf{LP$\rightarrow$MST:} An approximate inference approach
    based on an LP relaxation of \textbf{ILP-Dep}. As discussed
    in \S\ref{dep}, a maximum spanning tree is recovered
    from the output of the LP and greedily pruned in order to generate a
    valid integral solution while observing the imposed compression rate.
\item \textbf{ILP-Dep:} A version of the joint ILP of \newcite{thadani13a}
    without n-gram variables and corresponding features.
\item \textbf{DP+LP$\rightarrow$MST:} An approximate joint inference approach
    based on Lagrangian relaxation that uses \textbf{DP} for the maximum
    weight subsequence problem and \textbf{LP$\rightarrow$MST} for the
    maximum weight subtree problem.
%    A variation on the above approach
%    that attempts to approximate an integral solution to the
%    maximum-weight subtree subproblem in each dual iteration
%    by finding a maximum spanning tree in the subgraph of the
%    LP solution (cf.~\S\ref{dep}).
\item \textbf{DP+LP:} Another Lagrangian relaxation approach that pairs
    \textbf{DP}
    with the non-integral solutions from an LP relaxation of the
    maximum weight subtree problem (cf.~\S\ref{dep}).
%    Lagrangian relaxation that uses
%    dynamic programming for the maximum weight
%    subsequence subproblem (cf.~\S\ref{bigram}) and an LP relaxation to
%    recover fractional solutions to the
%    maximum weight subtree subproblem (cf.~\S\ref{dep}).
\item \textbf{ILP-Joint:} The full ILP from \newcite{thadani13a},
    which provides an upper bound on the performance of the proposed
    approximation strategies.
\end{packed_itemize}
The learning rate schedule for the Lagrangian relaxation approaches
was set as $\eta_i \triangleq \tau/(\tau + i)$,\footnote{
    $\tau$ was set to $100$ for aggressive subgradient updates.}
while the hyperparameter $\psi$ was tuned using the development
split of each corpus.\footnote{We were surprised to observe that
    performance improved significantly when $\psi$ was set
    closer to 1, thereby emphasizing token features in the
    dependency subproblem. The final values chosen were
    $\psi_\text{BN} = 0.9$ and $\psi_\text{NW}=0.8$.
}

\subsection{Results}
\label{fixed}
Tables~\ref{tab:bn} and \ref{tab:nw} summarize
the results from our compression experiments on the BN and NW corpora
respectively.
Starting with the n-gram approaches, the performance of \textbf{3-LM}
leads us to observe that the gains of supervised learning far outweigh
the utility of higher-order
n-gram factorization, which is also responsible for a significant
increase in wall-clock time.
In contrast, \textbf{DP}
is an order of magnitude faster than all other approaches
studied here %and performs nearly on par with the best joint inference
%approaches under n-gram evaluation metrics,
although it is not
competitive under parse-based measures such as RASP F$_1\%$ which is known
to correlate with human judgments of grammaticality~\cite{clarke06a}.

We were surprised by the strong performance of the dependency-based
inference techniques, which yielded results that approached the
joint model in both n-gram and parse-based measures. The exact
\textbf{ILP-Dep} approach halves the runtime of \textbf{ILP-Joint}
to produce compressions that have similar (although statistically
distinguishable) scores.
%are significant under Wilcoxon's signed rank test ($p < 0.05$).
Approximating dependency-based inference with \textbf{LP$\rightarrow$MST}
yields similar performance for a further halving of runtime;
however, the performance of this approach is notably worse.
%from that of
%the approximate joint inference techniques.

Turning to the joint approaches, the strong performance of \textbf{ILP-Joint}
is expected; less so is the
relatively high but yet practically reasonable runtime that it requires.
We note, however, that these ILPs are solved using a highly-optimized
commercial-grade solver that can utilize all CPU cores\footnote{16 cores in our experimental
    environment.} while our approximation approaches are implemented as
single-processed Python code without significant effort toward optimization.
Comparing the two approximation strategies shows a clear performance
advantage for \textbf{DP+LP} over \textbf{DP+LP$\rightarrow$MST}: the
latter approach entails slower inference due to the overhead of running
the Chu-Liu Edmonds algorithm at every dual update, and furthermore, the
error introduced by approximating an integral solution results in
a significant decrease in dependency recall.
In contrast, \textbf{DP+LP} directly optimizes the dual problem
by using the relaxed dependency solution to update Lagrange multipliers
and achieves the best performance on parse-based F$_1$
outside of the slower ILP approaches.
Convergence rates
also vary for these two techniques: \textbf{DP+LP} has a lower rate
of empirical convergence (15\% on BN and 4\% on NW) when compared to
\textbf{DP+LP$\rightarrow$MST} (19\% on BN and 6\% on NW).

Figure~\ref{fig:timing} shows the effect of input sentence length on inference
time and performance for \textbf{ILP-Joint} and \textbf{DP+LP} over the
NW test corpus.\footnote{Similar results were observed for the BN test corpus.}
The timing results reveal that the approximation strategy is consistently
faster than the ILP solver.
The variation in RASP F$_1$\% with input size
indicates the viability of a hybrid approach which could balance accuracy
and speed by using
\textbf{ILP-Joint} for smaller problems and \textbf{DP+LP} for larger ones.
%and that runtime
%gains tend to increase as the input size increases.
%Finally, although
%absolute F$_1$ for all measures tends to decline with larger inputs,
%there is no consistent trend in the relative performance of these approaches.

%the relative quality of system
%%output---indicated here by the difference in RASP
%%F$_1$\%---between
%output does not demonstrate
%an observable trend.


\section{Related Work}
\label{related}

Sentence compression is one of the better-studied text-to-text generation
problems % initially proposed by \newcite{dras97}
and has been observed to play a significant role in
human summarization~\cite{jing00a,jing00b}.
Most approaches to sentence compression are supervised
\cite{knight02,riezler03,turner05,mcdonald06a,unno06,galley07,nomoto07,cohn09,galanis10,ganitkevitch11,napoles11b,filippova13} following the release of
datasets such as the Ziff-Davis corpus~\cite{knight00} and the
Edinburgh compression corpora~\cite{clarke06a,clarke08}, although
unsupervised approaches---largely based on ILPs---have
also received consideration~\cite{clarke07,clarke08,filippova08a}.
Compression has also been used as a tool for document
summarization~\cite{daume02,zajic07,clarke07,martins09a,bergkirkpatrick11,woodsend12,almeida13,molina13,li13,qian13}, with recent work
formulating the summarization task as joint sentence extraction and
compression and often employing %inference approaches based on
ILP or Lagrangian relaxation.
Monolingual compression also faces many obstacles
common to decoding in machine translation, and
a number of approaches which have been proposed to combine
phrasal and syntactic models~\cite{huang07,rush11} \emph{inter alia}
offer directions for future research into
compression problems.

%\begin{figure}[t]%[htbp]
%\setlength{\abovecaptionskip}{-2pt}
%\setlength{\belowcaptionskip}{-4pt}
%\hspace{-6pt}
%\includegraphics[width=0.48\textwidth]{rasptime2.png}
%\caption{Effect of input size on
%    (a) inference time, and (b) the corresponding difference
%    in RASP F$_1$\% (\textbf{ILP-Joint} -- \textbf{DP+LP}) on the NW corpus.}
%\label{fig:timing}
%\end{figure}

\section{Conclusion}
\label{conclusion}
\vspace{-2pt}
We have presented approximate inference strategies to jointly compress
    sentences under bigram and dependency-factored objectives
    by exploiting the modularity of the task and considering the
    two subproblems in isolation.
%    While efficient dynamic programming
%    algorithms exist for bigram-based sentence compression,
%    generating dependency parses from general graphs is NP-hard.
%    We study approximate approaches to this problem using
%    LP relaxation and maximum spanning tree algorithms, yielding
%    approaches that can be combined with the efficient bigram-based
%    inference approach using Lagrange multipliers.
    Experiments show that one of these
    approximation strategies produces results comparable to
    a state-of-the-art integer linear program for the same
    joint inference task with a 60\% reduction
    in average inference time. %when compared to highly-optimized solvers.

%%Do not number the acknowledgment section.
%\section*{Acknowledgments}
%\vspace{-2pt}
%The authors are grateful to Ryan McDonald for helpful discussions
%and to the anonymous reviewers for their comments.
%This work was supported by the Intelligence Advanced
%Research Projects Activity (IARPA) via Department of Interior National
%Business Center (DoI/NBC) contract number D11PC20153.
%The U.S. Government is authorized to
%reproduce and distribute reprints for Governmental purposes
%notwithstanding any copyright annotation thereon.\footnote{
%%\textbf{Disclaimer:}
%The views and conclusions contained
%herein are those of the authors and should not be interpreted as necessarily
%representing the official policies or endorsements, either expressed or
%implied, of IARPA, DoI/NBC, or the U.S. Government.
%}

\bibliographystyle{acl.bst}
\bibliography{papers}

\end{document}
